% 使用 fontset=none 以关闭默认字体设置
\documentclass[fontset=none]{ctexart}

% 修改章节标题格式(演示)
\ctexset{section/format+=\sffamily}

% 修改页眉页脚(演示)
\usepackage{fancyhdr}
\fancyhf{}
\fancyhead[C]{\small 某某大学本科毕业论文(2019)}
\fancyfoot[C]{\small \thepage}
\pagestyle{fancy}

% 西文字体
\setmainfont{Times New Roman}
\setsansfont{Arial}  % Windows 下可使用类似的 Arial 字体
\setmonofont{Courier New}

% 中文字体(macOS)
\setCJKmainfont{SimSun}[AutoFakeBold, AutoFakeSlant]
\setCJKsansfont{SimHei}[AutoFakeBold, AutoFakeSlant]
\setCJKmonofont{FangSong}[AutoFakeBold, AutoFakeSlant]

% 定义单独的字体族和命令,关闭倾斜、加粗效果
\newCJKfontfamily[zhsong]\songti{SimSun}[BoldFont=*, ItalicFont=*, BoldItalicFont=*]
\newCJKfontfamily[zhhei]\heiti{SimHei}[BoldFont=*, ItalicFont=*, BoldItalicFont=*]
\newCJKfontfamily[zhkai]\kaishu{KaiTi}[BoldFont=*, ItalicFont=*, BoldItalicFont=*]
\newCJKfontfamily[zhfs]\fangsong{FangSong}[BoldFont=*, ItalicFont=*, BoldItalicFont=*]

% 重定义 \emph 和 \strong 的样式,参见 fontspec 宏包文档
\emfontdeclare{\kaishu\itshape}
\strongfontdeclare{\heiti\bfseries}

\begin{document}

\section{中文字体}

\subsection{文本标记}

\begin{center}
  \begin{tabular}{ccc}
    & \verb|\emph{...}| & \verb|\strong{...}| \\
    \hline
    文本 text & \emph{强调 emph}  & \strong{关键字 strong}
  \end{tabular}
\end{center}

\subsection{字体命令}

\begin{center}
  \begin{tabular}{c|cccc}
    & & \verb|\itshape| & \verb|\bfseries| & \verb|\itshape\bfseries| \\
    \hline
    \verb|\rmfamily| & \rmfamily 罗马体 roman & \rmfamily\itshape 倾斜 italic & \rmfamily\bfseries 加粗 bold & \rmfamily\itshape\bfseries 粗斜 bold-italic \\
    \verb|\sffamily| & \sffamily 无衬线 sans  & \sffamily\itshape 倾斜 italic & \sffamily\bfseries 加粗 bold & \sffamily\itshape\bfseries 粗斜 bold-italic \\
    \verb|\ttfamily| & \ttfamily 打字机 mono  & \ttfamily\itshape 倾斜 italic & \ttfamily\bfseries 加粗 bold & \ttfamily\itshape\bfseries 粗斜 bold-italic
  \end{tabular}
\end{center}

\subsection{更多字体命令}

\begin{center}
  \begin{tabular}{cccc}
    \verb|\songti|    & \verb|\heiti|   & \verb|\kaishu|   & \verb|\fangsong| \\
    \hline
    \songti 宋体 song & \heiti 黑体 hei & \kaishu 楷体 kai & \fangsong 仿宋 fang
  \end{tabular}
\end{center}

\end{document}